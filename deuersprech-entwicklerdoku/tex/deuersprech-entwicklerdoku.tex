\documentclass[a4paper]{article}

\usepackage[top=2.5cm, right=2cm, left=2cm, bottom=2cm]{geometry}
\usepackage{hyperref}

\begin{document}
    \begin{center}
        \LARGE
        Deuersprech-Kompilierer \\
        Dokumentation für Entwickler \\
    \end{center}
    
    \section{Projekt Bauen}
    Um das Projekt zu bauen, ist folgendes notwendig:
    \begin{itemize}
        \item JDK installiert und in Path Variable
        \item maven installiert und in Path Variable
        \item im Verzeichnis \texttt{deuersprech-maven/deuersprech-x.y/} den Befehl\\
        \\
         \texttt{\$ mvn clean compile assembly:single} \\
        \\
         ausführen
    \end{itemize}
    Danach sollte der Kompilierer als \texttt{.jar}-Datei im Verzeichnis \texttt{deuersprech-x.y/target} zum Ausführen bereit sein. \\
    Der Aufruf des Kompilierers erfolgt mit \\
    \\
    \texttt{\$ java -jar deuersprech-kompilierer.jar <Dateipfad zu Textdatei mit gültigem deuersprech>} \\

    \noindent
    Das Kompilat kann mit der JRE ausgeführt werden:\\
    \\
    \texttt{\$ java deuersprechAusstelle}

    \section{Projekt testen}
    Um die Testfälle auszuführen, ist folgendes notwendig
    \begin{itemize}
        \item JDK und maven installiert sowie in Path Variable 
        \item im Verzeichnis \texttt{deuersprech-maven/deuersprech-x.y/} den Befehl\\
        \\
         \texttt{\$ mvn test} \\
        \\   
    \end{itemize}
    Die Konsolenausgabe gibt Aufschluss darüber, wie die Tests verlaufen sind. 
    Ein detaillierteres Testprotokoll wird von TestNG in HTML generiert und ins Verzeichnis \texttt{deuersprech-maven/deuersprech-0.1/target/surefire-reports} kopiert.

    \section{Verwendete Bibliotheken/ Drittprogramme}
    \begin{itemize}
        \item \href{https://www.antlr.org/}{https://www.antlr.org} \\
            ANTLR wird genutzt, um Lexer und Parser aus der Grammatik zu generieren. 
        \item \href{http://jasmin.sourceforge.net/}{http://jasmin.sourceforge.net} \\
            Jasmin wird genutzt, um die JVM über eine Art Assembler-Sprache anzusprechen. Die Sprache, zu die der Deuersprech-Kompilierer zunächst übersetzt, ist der Befehlssatz von Jasmin. 
            Jasmin erstellt die ausführbare \texttt{.class}-Datei.
        \item \href{https://testng.org/doc/}{https://testng.org} \\
            TestNG wird als Testframework verwendet.
        \item \href{https://maven.apache.org/}{https://maven.apache.org} \\
            Maven wird als Build-Managent-Tool verwendet.
        \item \href{http://www.doxygen.nl/}{http://www.doxygen.nl} \\
            Doxygen wird genutzt, um mittels speziellen Kommentaren eine Dokumentation der Quelldateien zu generieren.
    \end{itemize}

    \section{Wichtige Komponenten}
    \subsection{Kontextfreie Grammatik}
    Die Grundlage des Kompilierers ist die kontextfreie Grammatik in 
    \texttt{deuersprech-maven/deuersprech-x.y/grammar}. In \texttt{Arithmetic.g4} wird die Syntax eines gültigen Deuersprech-Programmes definiert.
    Das Tool ANTLR generiert auf Grundlage der Grammatik Java-Code, welcher den Lexer sowie Teile des Parsers implementiert. Der Aufruf hierfür \\
    \\
    \texttt{\$ java -jar ../antlr-4.7.2-complete.jar -package de.deuersprech.parser \\ -o ../src/main/java/de/deuersprech/parser/ -no-listener -visitor ./Arithmetic.g4} \\
    \\
    ist im Verzeichnis \texttt{deuersprech-maven/deuersprech-x.y/grammar} auszuführen. Nach erfolgreichem Abschluss befinden sich die generierten Quelldateien im Verzeichnis \texttt{/deuersprech-x.y/src/main/java/de/deuersprech/parser}.

    \subsection{Ablauf des Deuersprech-Kompilieres}
    \noindent
    Die Main-Funktion des Deuersprech-Kompilierers erwartet einen Pfad zu einer Textdatei mit einem Deuersprech-Programm als Kommandozeilenargument.
    Die Main-Funktion startet Lexer und Parser und übergibt diesen die Deuersprech-Quelldatei. Der Parser gibt als Ergebnis einen abstrakten Syntaxbaum (intern: ParseTree) zurück. \\
    
    \noindent
    Die Klasse \texttt{de.deuersprech.compiler.MyArithmeticVisitor} implementiert einen Visitor, der den ParseTree im top-down-Prinzip rekursiv durchläuft und für jeden Knoten im Baum entsprechende Anweisungen für Jasmin generiert. Für jede Regel in der Grammtik muss eine visit-Methode implementiert werden.\\

    \subsection{Jasmin}
    \noindent
    Wurden alle Anweisungen zusammengetragen, wird schließlich Jasmin aufgerufen. 
    Jasmin generiert aus den Anweisungen eine \texttt{.class}-Datei, die ins aktuelle Verzeichnis geschrieben wird. \\
    Diese \texttt{.class}-Datei stellt das Kompilat des Deuersprech-Kompilierers da.

    
    \section{Doxygen}
    Die Java-Quelldateien sind mit Doxygen-Kommentaren versehen. Der Aufruf \\
    \\
    \texttt{\$ doxygen docyconfig} \\
    \\
    im Verzeichnis \texttt{deuersprech-maven/deuersprech-x.y/} generiert sowohl ein navigierbares HTML-Dokument als auch ein LaTeX-Dokument im Verzeichnis \texttt{deuersprech-maven/deuersprech-x.y/doxygen}. \\
    Damit kann leicht ein Überblick über die vorhandenen Klassen und Methoden geschaffen werden. \\

    \noindent
    Oben genanntes ist nur möglich, wenn doxygen installiert ist. \\



    \pagebreak
    \section{Implementierung des Deuersprech-Kompilierers}
    Hier folgt irgendwann mal eine detaillierte Schilderung, wie zulässige Deuersprech-Aussagen übersetzt werden. \\
\end{document}